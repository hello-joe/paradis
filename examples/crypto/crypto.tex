\documentclass[12pt]{article}

\usepackage{fancyvrb}
%%\usepackage{hyperref}

\RecustomVerbatimEnvironment
  {Verbatim}{Verbatim}
  {frame=single}

\title{Simple Cryptography in Erlang}

\author{Joe Armstrong}

\begin{document}

\maketitle

\tableofcontents

\section{Introduction}

In this note we present a number of {\sl essential} cryptographic algorithms.
Each of these algorithms solves a different cryptographic problem.
They can be used individually or combined to solve different problems.

We will give examples of the following:

\begin{itemize}
\item Computing a cryptographic hash of data.
\item A Public/Private key asymmetric cryptographic system.
\item A symmetric key system.
\item A secret sharing algorithm.
\end{itemize}

All the code accompanying this article is in the directory\\
\verb+https://github.com/joearms/paradis/tree/master/examples/crypto+

\section{Cryptographic hashes}



\begin{tabular}{|p{10cm}}
A cryptographic hash function is a hash function which is considered
practically impossible to invert, that is, to recreate the input data
from its hash value alone. These one-way hash functions have been
called "the workhorses of modern cryptography".[1] The input data is
often called the message, and the hash value is often called the
message digest or simply the digest.

The ideal cryptographic hash function has four main properties:


\begin{itemize}
  \item it is easy to compute the hash value for any given message
  \item it is infeasible to generate a message that has a given hash
  \item it is infeasible to modify a message without changing the hash
  \item it is infeasible to find two different messages with the same hash.
\end{itemize}

From: \verb+http://en.wikipedia.org/wiki/Cryptographic_hash_function+

\end{tabular}

SHA1 is one of the most commonly used cryptographic hash algorithms.
It produces a 120 bit hash of a data set.
SHA1 is part of the Erlang standard libraries.

There are two ways of calling it:

\begin{Verbatim}[frame=single]
digest1() ->
    crypto:hash(sha, "hello world").

digest2() ->
    S0 = crypto:hash_init(sha),
    S1 = crypto:hash_update(S0, "hello "),
    S2 = crypto:hash_update(S1, "world"),
    crypto:hash_final(S2).
\end{Verbatim}

The first example can be used when the data whose hash value is needed
is small.  The second where the data concerned is large. For example, if
we wanted to compare digital images of a few MBytes we could use
the first method, but to compute the SHA1 checksum of a GByte movie we
would use the second method.

\subsection{Applications of SHA1}

The single most important application of cryptographic hashing is in {\sl
  validation}. Two data sets can be considered identical if they have
the same SHA1 checksum.

{\sl Note: This is not a mathematical certainty. If we have more than $2^{120}$
  different files then two will have the same SHA1 checksum}

\section{Public Key Systems}

In a public key system two different keys are used. One key is used to
encrypt the data and a different key is used to decrypt the data.
Use of different keys is called {\sl Asymmetric Encryption}.
The RSA\footnote{Named after Don Rivest, Adi Shamir and Leonard Adleman.}  
algorithm makes use of two keys \verb+{A,N}+ and
\verb+{B,N}+.

Here's a simple example, first we generate two primes \verb+P+ and \verb+Q+:

\begin{verbatim}
> P=demo:make_prime(10).
214578232357
> Q=demo:make_prime(10).
10643821901
\end{verbatim}

Then we make the two keys:

\begin{verbatim}
> {A,B,N}=demo:make_key(P,Q).
{65537,149155912722954489473,2283932489039303450657}
\end{verbatim}

The modulus \verb+N+ is just \verb+P*Q+:

\begin{verbatim}
> P*Q.
2283932489039303450657
\end{verbatim}

To encode the integer \verb+1234+ (a secret), we compute:
$1234^A mod N$:

\begin{verbatim}
> C = lin:pow(1234,A,N). 
88048242822024428139
\end{verbatim}

To decrypt we use the other key:

\begin{verbatim}
> lin:pow(C,B,N). 
1234
\end{verbatim}

We could, of course, use the keys in the opposite order:

\begin{verbatim}
> C1 = lin:pow(1234,B,N).
96928412039852238659
> lin:pow(C1,A,N).       
1234
\end{verbatim}

\verb+make_key+ is like this:

\begin{Verbatim}[frame=single]
make_key(P, Q) ->
    N = P * Q,
    Phi = (P-1) * (Q-1),
    E = 65537,
    D = lin:inv(E, Phi),
    {E,D,N}.
\end{Verbatim}

\verb+inv(A, B)+ computes \verb+C+ such that \verb+A*C mod B = 1+

\subsection{Applications of RSA}

RSA is {\sl slow} and {\sl can only encrypt a small amount of data}
(ie some value less than \verb+N+). This is not a problem since we
typically use it to encrypt an SHA1 checksum (120 bits) or a short password.

The above algorithm is called ``text book RSA'' - in reality
the key has to be padded\footnote{See OAEP - ``Optimal Asymmetric Encryption padding.''}
to the N bits of the modulus.

To speed up modulo arithmetic we might use ``Montgomery reduction''
(ie computations module N are time consuming, so we do this modulo
$2^K$ which is easier, then do some transformation to compute modulo
$N$).

2048 bit modulos are considered secure\footnote{The world record is
  RSA-768 (2000 years on single code 2.2GHz AMD Opteron.}.


\section{Symmetric Encryption}

AES\footnote{The Advanced Encryption Standard.} is a set of encryption
  methods which use a {\sl symmetric encryption algorithm}. This means
  that the same key is used for encryption and decryption. Here's an
  example, making use of the code in \verb+lib1_aes.erl+.

\begin{Verbatim}[frame=single]
test1() ->
    Password = "hello",
    Plain = <<"this is plain text">>,
    Code = lib1_aes:encrypt(Password, Plain),
    io:format("Code=~p ~p~n",[Code,size(Code)]),
    Plain = lib1_aes:decrypt(Password, Code).
\end{Verbatim}

\subsection{Stream Encryption}

To encrypt a stream of data we can use a stream encryption algorithm.
Each iteration of the algorithm produces a new key to be used for the
next iteration of the algorithm, for example, using RC4 we can write:

\begin{Verbatim}[frame=single]
test1() ->
    K1 = crypto:stream_init(rc4, "secret password"),
    {K2, C1} = crypto:stream_encrypt(K1, "hello "),
    {K3, C2} = crypto:stream_encrypt(K2, "world"),
    %% ...
    {S1, M1} = crypto:stream_decrypt(K1, C1),
    {S2, M2} = crypto:stream_decrypt(S1, C2),
    {M1, M2}.
\end{Verbatim}

Running this:

\begin{Verbatim}[frame=single]
> crypto_examples:test1().
{<<"hello ">>,<<"world">>}
\end{Verbatim}

\section{Secret Sharing}

Shamir's Secret sharing algorithm provides a method of sharing a
secret into $N$ parts such that any $K$ parts can recover the
secret. Here $K <= N$ and $N > 0$.

The implementation in this directory is due to Robert Newson and was
published at \verb+https://github.com/rnewson/shamir/+

As an example, suppose we want to share the secret \verb+hello+ using
seven shares, so that any three shares unlock the secret. We can
generate the shares like this:

\begin{Verbatim}[frame=single]
> L=shamir:share(<<"hello">>, 3,7).                                                
[{share,3,1,<<206,145,84,97,217>>},
 {share,3,2,<<229,208,230,155,102>>},
 {share,3,3,<<67,36,222,150,208>>},
 {share,3,4,<<56,15,170,12,128>>},
 {share,3,5,<<158,251,146,1,54>>},
 {share,3,6,<<181,186,32,251,137>>},
 {share,3,7,<<19,78,24,246,63>>}]
\end{Verbatim}

Using shares 1 2 and 5 we can reconstruct the secret as follows:

\begin{Verbatim}[frame=single]
> shamir:recover([lists:nth(1,L),lists:nth(2,L),lists:nth(5,L)]).
<<"hello">>
\end{Verbatim}

The algorithm fails if we don't give it three different shares:

\begin{Verbatim}[frame=single]
> shamir:recover([lists:nth(1,L),lists:nth(2,L),lists:nth(2,L)]).
** exception error: no function clause matching
   shamir:recover(3,[{1,10},{2,199}]) (shamir.erl, line 50)
     in function  shamir:'-recover/1-lc$^2/1-1-'/2 (shamir.erl, line 48)
     in call from shamir:recover/1 (shamir.erl, line 48)
\end{Verbatim}

Again the shared secret should be a password or digest that unlocks or
validates the content of another file.

\section{Putting it all together}

This note has shown a number of different cryptographic algorithms.
To make a secure system we combine these in various ways. In building a system
we combine the cryptographic code with code to read and write files and sockets
and to serialize and deserialize data.

We can use the \verb+gen_tcp+ to read and write TCP sockets, and
\verb+term_to_binary+ and \verb+binary_to_term+ to turn Erlang terms into
data packets which can be stored in files or transmitted in data packets.




\end{document}

