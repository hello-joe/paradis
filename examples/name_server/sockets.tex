\documentclass[12pt]{article}

\usepackage{fancyvrb}
%%\usepackage{hyperref}

\RecustomVerbatimEnvironment
  {Verbatim}{Verbatim}
  {frame=single}

\title{Fun with sockets}

\author{Joe Armstrong}

\begin{document}

\maketitle

\tableofcontents

\section{Introduction}

In this note we make an extremely simple model of how name servers,
resolvers, clients and servers interact. This is the primary model for
interaction, used by billions of computers on the Internet.

The goal is to make our programs as simple as possible so we can see
how the bits fit together. 

We need four programs:

\begin{itemize}
\item \verb+triv_tcp_name_server+ is a name server
\item \verb+triv_tcp_resolver+ is a resolver
\item \verb+triv_tcp_fac_server+ is a factorial server
\item \verb+triv_tcp_fac_client+ is a factorial client
\end{itemize}

Before starting I'll just remind you of how to build a 
parallel server in Erlang. This will be the model for the
two servers in this note.

\section{The simplest possible parallel server}

The code for a parallel server  is from my Erlang book, page 271, but with a few
modifications.  The code below is illustrative rather than runnable.
\verb+options+ must be replaced by a list or options
and \verb+do_something+ must be replaced by some code.

\VerbatimInput[frame=single]{par_server.erl}

\section{A Trivial TCP Name Server}

A name server is a essentially a Key-Value database 
that can be accessed through a message passing interface.

We can add Key-Value pairs to the database, and we can look up the
value of a Key and retrieve the associated value.

Our server follows the pattern of a parallel server given in the previous section
and is as follows:

\VerbatimInput[frame=single]{triv_tcp_name_server.erl}

The program has two processing loops. \verb+connection_loop+ handles
the connection to a single client and lives for as long as a client is
connected to the name server.  \verb+name_server_loop+ is a single
process that lives forever.\footnote{well at least until the server is
  stopped.}

Accessing the server is done through a client program called a resolver.

\section{A Trivial TCP Resolver}

A resolver runs an a client machine and is used to access the services
provided by a name server.

Here's our trivially simple TCP resolver:


\VerbatimInput[frame=single]{triv_tcp_resolver.erl}

The resolver needs to know the hostname and port where the server is running.
It can send two messages to the server. 
The \verb+add+ message associates a key with a value. 
\verb+lookup+ looks up the value of a key.

\section{Trivial Factorial Server}

The factorial server follows the same pattern as the original parallel server
and in addition it registers its name with the name server.

\VerbatimInput[frame=single]{triv_tcp_fac_server.erl}

\section{Trivial TCP Factorial Client}

The factorial client find the address of a factorial server by querying the 
names server then directs a query to the factorial server.

\VerbatimInput[frame=single]{triv_tcp_fac_client.erl}

\section{A Trial Session}

Compile everything then create three windows:

\begin{Verbatim}
In Window 1
> triv_tcp_name_server:start()
Starting name server on port 6000

In Window 2
> triv_tcp_fac_server:start().
Starting factorial server on port 6010

And in window 1 we see
Adding:"fac" => {"localhost",6010}

In window 3
> triv_tcp_fac_client:fac(10).
In window 1 we see
Lookup:"fac"
Reply:{value,{"fac",{"localhost",6010}}}

And in window 3
{ok,{faq_response,3628800}}
\end{Verbatim}

\section{Reality}

The programs here are very much simplified compared the  versions found in
a real system. Here are some of the features that are missing:

\begin{itemize}
\item A real name server needs a security mechanism. Our name server allows
anybody to register a key-value pair on the server.

\item A real name server has a time-to-live parameter for the data in the store.

\item Real name servers recursively query other name servers if they do not know
the value of key

\item A real resolver maintains a local cache of previous key-value queries. The
cached items have a time-to-live parameter obtained from the name servers.

\item Real resolvers may query several different name servers either
serially or in parallel.

\item We use \verb+{packet,4}+ to frame the TCP packets and
  \verb+term_to_binary+ and the inverse \verb+binary_to_term+ to
  encode and decode packets. Real name servers use a binary packet
  format.

\item There is no mechanism to remove a name from the name server if the
  server associated with the name dies.

\item Consistency issues are not addressed. There are times when the 
system is in an inconsistent state.

\item There are no timeouts in the code.


\end{itemize}


\end{document}
